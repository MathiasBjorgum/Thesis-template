
Et kapittel vil ofte bli etterfulgt av en kort ingress. Her beskriver vi hva dette innledningskapittelet skal handle om. Som regel vil det være en aktualisering av tema og beskrivelse av problemstilling. I denne templaten har jeg valgt å ta med utsnitt fra min bacheloroppgave for å visualisere hvordan et større dokument kan se ut.

\section{Bakgrunn for oppgaven}
Da vi skulle velge en bedrift å skrive semesteroppgave om ønsket vi først og fremst å velge en bedrift som er norsk, nytenkende og med en spennende fremtid. Valget falt på Kahoot!.

Kahoot! har blitt et teknologisk hjelpemiddel for læring de seneste årene med et stort marked. Konseptet har raskt blitt populært, og blir benyttet av flere aktører som privatpersoner, bedrifter og skoler. For lærere og studenter er Kahoot! gratis, mens bedrifter må betale for årlige abonnement. Et interessant spørsmål knyttet til dette er derfor hvordan Kahoot! tjener penger, og om det er lønnsomt i et langsiktig løp. Ved en kjapp titt på regnskapstallene så vi at Kahoot! ikke har tjent penger siden oppstarten, noe vi synes var merkelig tross populariteten. Derfor falt valg av oppgave på en lønnsomhetsanalyse av Kahoot!.

\section{Tema og problemstilling}
\label{sec:innledning:problemstilling}
Vår problemstilling er følgende:
\say{\emph{Er Kahoot! lønnsom?}}. 
For å svare på denne problemstillingen tar vi utgangspunkt i perioden 2015-2020. Vi vil komme tilbake til konkretisering av problemstilling i \verb!\autoref{sec:metode}! av denne oppgaven.

\section{Oppgavens struktur}
Her er det naturlig å beskrive hvordan oppgaven vil utvikle seg videre. Det er gunstig å bruke \verb!\autoref{}! for å gjøre det enkelt å klikke seg rundt.


\section{Presentasjon av bedriften}

Kahoot! AS er en norsk virksomhet som jobber med e-læring. På deres hjemmesider står det at målet deres er å gjøre læring fantastisk (\cite{About_Us}). Dette ønsker de å få til gjennom spillbasert og interaktiv læring. Selskapet ble grunnlagt i 2012 av studenter ved NTNU i samarbeid med professor Alf Inge Wang og entreprenør Åsmund Furseth.

Selskapet lanserte en privat beta i mars 2013, og åpnet den til en offentlig beta allerede i september samme år. Etter det har Kahoot! opplevd en stadig økende brukerbase både i skolesektoren og i samfunnet for øvrig.

Kahoot! AS mener at vi lærer nye ting gjennom lek og nysgjerrighet og at gjennom å kombinere dette vil de kunne frigjøre det potensialet for læring som finnes inni hver og en av oss, uansett alder, emne eller evne. Gründerene av Kahoot! drives av å frigjøre læringspotensialet hos alle, og derfor har de tatt på seg oppdraget for å gjøre læring fantastisk. Dette gjør de ved å bygge den mest engasjerende og kraftfulle opplevelsen for læring, og gjennom visjonen til selskapet som er å bygge Den verdensledende plattformen for læring.

Fra 2019 har Kahoot! ekspandert og kjøpt opp selskaper i samme bransje. Dette gjelder særlig oppkjøpet av DragonBox og Poio. Dette var eksisterende applikasjoner som skulle lære barn henholdsvis matematikk og leseferdigheter.


Så langt ser det ut til at Kahoot! er i ferd med å bli en stor suksess. Vi kan blandt annet se dette gjennom at 97\% av bedrifter på Fortune 500 benytter Kahoot! til diverse formål. Siden lanseringen har mer enn 4 milliarder mennesker, fra over 200 land, spilt på Kahoot! sin plattform.